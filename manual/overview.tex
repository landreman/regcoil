\chapter{Overview}

This program is an implementation of the \regcoil~algorithm described in \cite{regcoilPaper},
and was used for the calculations in \cite{regcoilPaper}.
This paper is available in the Git repository for this program.
In addition, this program can read output files from the \nescoil~code \cite{nescoil},
processing the results to compute quantities like the current density
and residual magnetic field normal to the target plasma surface.

%In the various variable names in the code and output file, `r' refers
%to the position vector, not to a radius.  In various arrays with a
%dimension of length 3, this dimension always corresponds to Cartesian
%coordinates $(x,y,z)$.

%The `normal' quantities in the code and output file refer to the
%surface normal vector N = (dr/dv) cross (dr/du) as in the NESCOIL
%paper. Note that this vector does not have unit magnitude.



\section{Required libraries}

\begin{itemize}

\item {\ttfamily NetCDF} (for writing the output file)
\item {\ttfamily BLAS} (for matrix multiplication)
\item {\ttfamily LAPACK} (for solving linear systems and for singular value decomposiiton)

\end{itemize}

Most of these libraries will be available on any high-performace computing system. {\ttfamily BLAS} and {\ttfamily LAPACK}
are available on Apple Mac computers (as part of the Accelerate framework) if you install Xcode through the App store.

If {\ttfamily OpenMP} is available, calculations with the code are parallelized.
The plotting and testing functions use \python,
{\ttfamily numpy}, and {\ttfamily scipy}.
The plotting routines \regcoilPlot~and {\ttfamily compareRegcoil} use {\ttfamily matplotlib}.

\section{Cloning the repository}

The source code for \regcoil~is hosted in a {\ttfamily git} repository at
\url{https://github.com/landreman/regcoil}.
You obtain the \regcoil~source code by cloning the repository. This requires several steps.

\begin{enumerate}
\item Create an account on \url{github.com}, and sign in to {\ttfamily github}.
\item Click the icon on the top right to see the drop-down menu of account options, and select the ``Settings'' page.
\item Click on ``SSH and GPG keys'' on the left, and add an SSH key for the computer you wish to use. To do this, you may wish to read see the ``generating SSH keys'' guide which is linked to from that page: \url{https://help.github.com/articles/connecting-to-github-with-ssh/}
\item From a terminal command line in the computer you wish to use, enter\\
{\ttfamily git clone git@github.com:landreman/regcoil.git}\\
 to download the repository.
\end{enumerate}

Any time after you have cloned the repository in this way, you can download future updates to the code by entering {\ttfamily git pull} from any subdirectory within your local copy.



\section{Parallelization}

The code does not use {\ttfamily MPI}, and so it runs on a single computing node.  However, it is possible to use multiple threads
on the node to accelerate computations.  The multi-threaded parallelization is done in part using {\ttfamily OpenMP}
and in part using a multi-threaded {\ttfamily BLAS} routine. Typically the number of threads is set by
setting the environment variable {\ttfamily OMP\_NUM\_THREADS}.

%The slowest steps in \regcoil~are typically the assembly of the
%matrices described in appendix B of \cite{regcoilPaper}.  For each of these two matrices, there are two slow steps.
%The first is computation of the magnetic dipole formula between each pair of points
%on the two toroidal surfaces.  The loop for this computation is parallelized using {\ttfamily OpenMP}.
%The other slow step is the integration of this result against Fourier modes,
%which is done using matrix multiplication with the {\ttfamily BLAS} subroutine {\ttfamily DGEMM}.
%To parallelize this step you can link \regcoil~with a multi-threaded  {\ttfamily BLAS} library,
%such as the Intel Math Kernel Library (MKL).

\section{\ttfamily make test}

To test that your \regcoil~executable is working, you can run {\ttfamily make test}.  Doing so will run
\regcoil~for some or all of the examples in the {\ttfamily examples/} directories.
After each example completes, several of the output quantities
will be checked, using the
{\ttfamily tests.py} script in the example's directory.
The {\ttfamily make test} feature is very useful when making changes to the code, since it allows you to check
that your code modifications have not broken anything and that previous results
can be recovered.

If you run {\ttfamily make retest},
no new runs of \regcoil~will be performed, but the {\ttfamily tests.py} script
will be run on any existing output files in the \path{/examples/} directories.

\section{Units}

As in \vmec, all of \regcoil's input and output parameters use SI units: meters, Teslas, Amperes, and combinations thereof.

\section{Plasma current}

If there is current inside the plasma, then this current will contribute to the magnetic field normal
to the target plasma surface, which the coils must cancel. The contribution of plasma current to the normal field
is not computed directly by \regcoil, but it can be computed using the \bnorm~code which is 
often distributed with \vmec.  You can then set \parlink{load\_bnorm}={\ttfamily .true.} and specify \parlink{bnorm\_filename}
in the \regcoil~input namelist to load the \bnorm~results into \regcoil.

\section{Matlab version}

Both \fortran~and \matlab~versions of \regcoil~are included in the repository.  The \matlab~version is
contained in the file {\ttfamily regcoil.m}. For normal
use you will want to use the \fortran~version, since it is much faster.  The \matlab~version was originally
written as a check of the \fortran~version, to verify that two independent implementations of the algorithm
in different languages give identical results.  The \matlab~version reads in an output file from the \fortran~version
and verifies that each significant variable is identical.  A few of the features in the \fortran~version
are not available in the \matlab~version.

\section{Plotting results}

The python program \regcoilPlot~will display many of the output quantities from a single \regcoil~calculation.
Results from multiple \regcoil~calculations can be compared using the python program {\ttfamily compareRegcoil}.

You can also make a 3D figure of the shapes of discrete coils using the \matlab~program {\ttfamily m20160811\_01\_plotCoilsFromRegcoil.m}.
Two different sets of coils can be plotted together using the \matlab~program {\ttfamily m20160811\_02\_compare2CoilsetsFromRegcoil.m}.
The latter program was used to generate the figure on the cover of this manual.

\section{Cutting coils}

Once a suitable current potential has been computed with \regcoil, you can `cut' discrete coils
using the python script {\ttfamily cutCoilsFromRegcoil}. You can run this script with no arguments
to see a list of the input parameters. This script will generate a coils file suitable for input to the
{\ttfamily makegrid} code (distributed with \vmec), which in turn generates an mgrid file used as input for free-boundary \vmec.

\section{Output quantities}

The output variables are documented using metadata (the `{\ttfamily long\_name}' attribute)
in the netCDF output files {\ttfamily regcoil\_out.<extension>.nc}.
To view the available output variables, their annotations, and their values, you can run
{\ttfamily ncdump regcoil\_out.<extension>.nc | less} from the command line.
Some of the most commonly used output quantities which can be found in the netCDF file are 
{\ttfamily lambda},
{\ttfamily chi2\_B},
{\ttfamily chi2\_K},
{\ttfamily max\_Bnormal},
{\ttfamily max\_K},
{\ttfamily chi2\_B\_target},
and
{\ttfamily current\_potential}.

\section{Questions, Bugs, and Feedback}

We welcome any contributions to the code or documentation.
For write permission to the repository, or to report any bugs, provide feedback, or ask questions, contact Matt Landreman at
\href{mailto:matt.landreman@gmail.com}{\nolinkurl{matt.landreman@gmail.com} }






